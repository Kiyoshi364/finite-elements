\documentclass{pssbmac}

%%%%%%%%%%%%%%%%%%%%%%%%%%%%%%%%%%%%%%%%%%%%%%%%%%%%%%%%%%%%%%%%%%%%%%%%
%% POR FAVOR, NÃO FAÇA MUDANÇAS NESSE PADRÃO QUE ACARRETEM  EM
%% ALTERAÇÃO NA FORMATAÇÃO FINAL DO TEXTO
%%%%%%%%%%%%%%%%%%%%%%%%%%%%%%%%%%%%%%%%%%%%%%%%%%%%%%%%%%%%%%%%%%%%%%%%

%%%%%%%%%%%%%%%%%%%%%%%%%%%%%%%%%%%%%%%%%%%%%%%%%%%%%%%%%%%%%%%%%%%%%%%%
% POR FAVOR, ESCOLHA CONFORME O CASO
%%%%%%%%%%%%%%%%%%%%%%%%%%%%%%%%%%%%%%%%%%%%%%%%%%%%%%%%%%%%%%%%%%%%%%%%
\usepackage[brazil]{babel} % texto em Português
%\usepackage[english]{babel} % texto em Inglês

%\usepackage[latin1]{inputenc} % acentuação em Português ISO-8859-1
\usepackage[utf8]{inputenc} % acentuação em Português UTF-8
%%%%%%%%%%%%%%%%%%%%%%%%%%%%%%%%%%%%%%%%%%%%%%%%%%%%%%%%%%%%%%%%%%%%%%%%


%%%%%%%%%%%%%%%%%%%%%%%%%%%%%%%%%%%%%%%%%%%%%%%%%%%%%%%%%%%%%%%%%%%%%%%%
%% POR FAVOR, NÃO ALTERAR
%%%%%%%%%%%%%%%%%%%%%%%%%%%%%%%%%%%%%%%%%%%%%%%%%%%%%%%%%%%%%%%%%%%%%%%%
\usepackage[T1]{fontenc}
\usepackage{float}
\usepackage{graphics}
\usepackage{graphicx}
\usepackage{epsfig}
\usepackage{indentfirst}
\usepackage{amsmath, amsfonts, amssymb, amsthm}
\usepackage{url}
\usepackage{csquotes}
% Ambientes pré-definidos
\newtheorem{theorem}{Theorem}[section]
\newtheorem{lemma}{Lemma}[section]
\newtheorem{proposition}{Proposition}[section]
\newtheorem{definition}{Definition}[section]
\newtheorem{remark}{Remark}[section]
\newtheorem{corollary}{Corollary}[section]
\newtheorem{teorema}{Teorema}[section]
\newtheorem{lema}{Lema}[section]
\newtheorem{prop}{Proposi\c{c}\~ao}[section]
\newtheorem{defi}{Defini\c{c}\~ao}[section]
\newtheorem{obs}{Observa\c{c}\~ao}[section]
\newtheorem{cor}{Corol\'ario}[section]

% ref bibliográficas
\usepackage[backend=biber, style=numeric-comp, maxnames=50]{biblatex}
\addbibresource{refs.bib}
\DeclareTextFontCommand{\emph}{\boldmath\bfseries}
\DefineBibliographyStrings{brazil}{phdthesis = {Tese de doutorado}}
\DefineBibliographyStrings{brazil}{mathesis = {Disserta\c{c}\~{a}o de mestrado}}
\DefineBibliographyStrings{english}{mathesis = {Master dissertation}}
%%%%%%%%%%%%%%%%%%%%%%%%%%%%%%%%%%%%%%%%%%%%%%%%%%%%%%%%%%%%%%%%%%%%%%%%


\begin{document}

%%%%%%%%%%%%%%%%%%%%%%%%%%%%%%%%%%%%%%%%%%%%%%%%%%%%%%%%%%%%%%%%%%%%%%%%
% TÍTULO E AUTORAS(ES)
%%%%%%%%%%%%%%%%%%%%%%%%%%%%%%%%%%%%%%%%%%%%%%%%%%%%%%%%%%%%%%%%%%%%%%%%

\title{Instruções para Submissão de Resumos para o CNMAC}

\author{
    {\large Sandra M. C. Malta}\thanks{autora1@email}, {\large Patricia R. Fortes}\thanks{autora2@email}\\
    {\small LNCC, Petrópolis, RJ} \\
    {\large Mateus Bernardes}\thanks{autor3@email}  \\
    {\small DAMAT/UTFPR, Curitiba, PR} \\
}
\criartitulo
%%%%%%%%%%%%%%%%%%%%%%%%%%%%%%%%%%%%%%%%%%%%%%%%%%%%%%%%%%%%%%%%%%%%%%%%


%%%%%%%%%%%%%%%%%%%%%%%%%%%%%%%%%%%%%%%%%%%%%%%%%%%%%%%%%%%%%%%%%%%%%%%%
% TEXTO
%%%%%%%%%%%%%%%%%%%%%%%%%%%%%%%%%%%%%%%%%%%%%%%%%%%%%%%%%%%%%%%%%%%%%%%%

Este é o padrão (formato \LaTeX{} apenas) para a submissão de trabalhos da Categoria 1 do CNMAC, destinados à divulgação de pesquisas em andamento, com resultados preliminares, e pesquisas em nível de Iniciação Científica. \emph{Nesta categoria, os trabalhos devem ser submetidos em Português ou Inglês, em forma de resumo de, no máximo, duas páginas, incluindo-se as referências bibliográficas.} Os \emph{trabalhos submetidos} que \emph{não estiverem de acordo com o formato} apresentado por esse padrão \emph{serão rejeitados} pelo Comitê Editorial do evento, sem análise do mérito científico.

Equações inseridas no resumo devem ser enumeradas sequencialmente e à direita no texto, por exemplo
\begin{equation}
\frac{\partial u}{\partial t}-\Delta u = f, \quad  \mathrm{em} \; \Omega. \label{Calor}
\end{equation}
Consulte o arquivo \verb!.tex! para mais detalhes sobre o código-fonte gerador da equação \eqref{Calor}.

Tendo em vista tratar-se de um resumo, sugere-se evitar a inserção de seções, tabelas e figuras. Caso necessária, a inserção de tabela deve ser feita com o ambiente \verb!table!, sendo enumerada, disposta horizontalmente centralizada, próxima de sua referência no texto, e a legenda imediatamente acima dela. Por exemplo, consulte a Tabela \ref{tabela01}.

\begin{table}[H]
\caption{ {\small Categorias dos trabalhos.}}
\centering
\begin{tabular}{ccc}
\hline
Categoria do trabalho  & Número de páginas & Tipo do trabalho\\ \hline
1          & 2  & $A$, $B$ e $C$    \\
2          & entre 5 e 7  & apenas $C$ \\
\hline
\end{tabular}\label{tabela01}
\end{table}

A inserção de figura deve ser feita com o ambiente \verb!figure!, ela deve estar enumerada, disposta horizontalmente centralizada, próxima de sua referência no texto, e legenda imediatamente abaixo dela. \emph{Quando não própria, deve-se indicar/referências a fonte.} Consulte a Figura \ref{figura01}.

\begin{figure}[H]
\centering
\includegraphics[width=.475\textwidth]{ex_fig}
\caption{ {\small Exemplo de imagem. Fonte: indicar.}}
\label{figura01}
\end{figure}

As referências bibliográficas devem ser inseridas conforme especificado neste padrão, sendo que serão automaticamente geradas em ordem alfabética pelo sobrenome do primeiro autor. Este {\it template} fornece suporte para a inserção de referências bibliográficas com o Bib\LaTeX{}. Os dados de cada referência do trabalho devem ser adicionados no arquivo \verb+refs.bib+ e a indicação da referência no texto deve ser inserida com o comando \verb+\cite+. Seguem alguns exemplos de referências: livro \cite{Boldrini}, artigos publicados em periódicos \cite{Contiero,Cuminato}, capítulo de livro \cite{daSilva}, dissertação de mestrado \cite{Diniz}, tese de doutorado \cite{Mallet}, livro publicado dentro de uma série \cite{Gomes}, trabalho publicado em anais de eventos \cite{Santos}, {\it website} e outros \cite{CNMAC}. Por padrão, os nomes de todos os autores da obra citada aparecem na bibliografia. Para obras com mais de três autores, é também possível indicar apenas o nome do primeiro autor, seguido da expressão et al. Para implementar essa alternativa, basta remover ``\verb+,maxnames=50+'' do comando correspondente do código-fonte. Sempre que disponível forneça o DOI, ISBN ou ISSN, conforme o caso.

\section*{Agradecimentos (opcional)}
Seção reservada aos agradecimentos dos autores, caso for pertinente. Por exemplo, agradecimento a fomentos. Se os autores optarem pela inclusão de Agradecimentos, a palavra ``(opcional)'' deve ser removida do título da seção.


%%%%%%%%%%%%%%%%%%%%%%%%%%%%%%%%%%%%%%%%%%%%%%%%%%%%%%%%%%%%%%%%%%%%%%%%
% REFS BIBLIOGRÁFICAS
% POR FAVOR, NÃO ALTERAR
%%%%%%%%%%%%%%%%%%%%%%%%%%%%%%%%%%%%%%%%%%%%%%%%%%%%%%%%%%%%%%%%%%%%%%%%
\printbibliography
%%%%%%%%%%%%%%%%%%%%%%%%%%%%%%%%%%%%%%%%%%%%%%%%%%%%%%%%%%%%%%%%%%%%%%%%

\end{document}
