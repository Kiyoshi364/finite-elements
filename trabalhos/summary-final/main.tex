\documentclass{pssbmac}

% { Template

\usepackage[brazil]{babel} % texto em Português
\usepackage[utf8]{inputenc} % acentuação em Português UTF-8

\usepackage[T1]{fontenc}
\usepackage{float}
\usepackage{graphics}
\usepackage{graphicx}
\usepackage{epsfig}
\usepackage{indentfirst}
\usepackage{amsmath, amsfonts, amssymb, amsthm}
\usepackage{url}
\usepackage{csquotes}

\usepackage[backend=biber, style=numeric-comp, maxnames=50]{biblatex}
\addbibresource{refs.bib}
\DeclareTextFontCommand{\emph}{\boldmath\bfseries}
\DefineBibliographyStrings{brazil}{phdthesis = {Tese de doutorado}}
\DefineBibliographyStrings{brazil}{mathesis = {Disserta\c{c}\~{a}o de mestrado}}
\DefineBibliographyStrings{english}{mathesis = {Master dissertation}}

% Template }

\usepackage{svg}
\svgpath{{images/}}

\newcommand{\titlelinebreak}{\texorpdfstring{\\}{ }}
\newcommand{\templauthoremail}[2]{{\large #1}\thanks{#2}}
\newcommand{\templlocation}[1]{{\small #1}}

\title{Usando valores pré-computados
    para economizar recursos
    na construção de matrizes
    para Método de Elementos Finitos
    com a Linguagem de Programação Julia
}

\author{
    \templauthoremail{Daniel K. Hashimoto V. de Andrade}{dkhashimoto@ic.ufrj.br} \\
    \templlocation{UFRJ, Rio de Janeiro, RJ} \\
}

\begin{document}

\criartitulo

Muitos fenômenos do mundo real
são modelados por sistemas de equações diferenciais.
Alguns desses sistemas possuem solução analítica,
mas muitos deles não.
Uma alternativa para esse problema
é buscar soluções discretas
próximas o suficiente da solução real
com a ajuda de um computador.
O Método de Elementos Finitos
segue essa ideia,
sendo capaz de resolver
uma grande variedade de classes de sistemas de equações.

Resolver um sistema de equações diferenciais
com um refinamento adequado
pode requisitar altos custos computacionais,
tanto em tempo quanto em memória.
Dessa forma,
uma solução refinada o suficiente
pode acabar sendo inviável de ser computada,
por demorar muito tempo ou
simplesmente não caber na memória do computador.
Por isso,
temos a preocupação de otimizar implementações
desse método para que
seja possível encontrar soluções
mais refinadas e em menos tempo.

A arquitetura geral de uma implementação
do Método de Elementos Finitos
já é bem estabelecida
e geralmente segue os seguintes passos:
particionar o espaço em elementos finitos;
escolher funções base para representar
a solução como combinação linear delas;
montar matrizes que relacionam
as restrições das funções base com a solução,
formando um sistema linear;
e resolver o sistema linear.
Escolhas espertas nos dois primeiros passos
podem simplificar
tanto a montagem quanto a solução do sistema linear.

Nesse trabalho,
damos foco na construção das matrizes
que representam o sistema linear.
E propomos estratégias
para construir essas matrizes
de uma forma que economize
os recursos computacionais tempo e memória.
As implementações propostas
se baseiam na ideia de pré-calcular
valores que se espera que serão utilizados várias vezes
e usar uma abstração de iteradores,
que a linguagem de programação Julia
disponibiliza.
E então,
testamos implementações em Julia
dessas propostas com benchmarks.

% \begin{figure} \centering
%     \includesvg[width=0.5\textwidth]{vec-min_time}
%     \caption{Implementações: Tempo por número de elementos}
% \end{figure}

\section*{Agradecimentos}

% A Portaria nº 206/2018 - CAPES
% pede para usar exatamente a frase a seguir:
O presente trabalho foi realizado com apoio da
Coordenação de Aperfeiçoamento de Pessoal de Nível Superior
-- Brasil (CAPES) --
Código de Financiamento~001.

% This study was financed in part by the
% Coordenação de Aperfeiçoamento de Pessoal de Nível Superior
% -- Brasil (CAPES) --
% Finance Code~001.

% { Template
\printbibliography
% Template }
\end{document}
